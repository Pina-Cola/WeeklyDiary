\documentclass[a4, 11pt]{scrartcl}
%----------------------------------------
% German language
% \usepackage[ngerman]{babel}
%----------------------------------------
% Input in UTF8 accepted
%\usepackage[utf8]{inputenc}
%----------------------------------------
% Math-packages
\usepackage{mathtools}
\usepackage{amsmath}
\usepackage{amssymb}
%----------------------------------------
% Design choices (?)
%\usepackage{scrlayer-scrpage}
%\pagestyle{scrheadings}
% \clearscrheadfoot
%----------------------------------------
% Footnotes in tables
%\usepackage{tablefootnote}
%----------------------------------------
%----------------------------------------
% no parindent
\setlength{\parindent}{0em}
%----------------------------------------
% different gap between paragraphs
%\setlength{\parskip}{1.3ex}
%----------------------------------------
% spacing between lines
%\usepackage[onehalfspacing]{setspace}
%----------------------------------------
% Hurenkinder und Schusterjungen vermeiden
\clubpenalty = 10000
\widowpenalty = 10000
\displaywidowpenalty = 10000
%----------------------------------------
% Hyperref package
\usepackage{hyperref}
\hypersetup{
	colorlinks=true,
	linkcolor=black,
	filecolor=black,      
	urlcolor=OliveGreen,
	citecolor=black
}
%----------------------------------------
% Geometry package
\usepackage{geometry}
\geometry{
	paper=a4paper, % Change to letterpaper for US letter
	top=3cm, % Top margin
	bottom=3cm, % Bottom margin
	left=2cm, % Left margin
	right=3cm, % Right margin
	%showframe, % Uncomment to show how the type block is set on the page
}



\usepackage{float}

\usepackage{array, booktabs}
\usepackage{graphicx}
\usepackage{tikz}
\graphicspath{ {./assets/} }
\usepackage[dvipsnames, table]{xcolor}
\usepackage{sectsty}
\sectionfont{\color{OliveGreen}}

\newcommand{\foo}{\color{OliveGreen}\makebox[0pt]{\textbullet}\hskip-0.5pt\vrule width 1pt\hspace{\labelsep}}

\usepackage{longtable}

\usepackage{wasysym}


\usepackage[framemethod=TikZ]{mdframed}

% Style %
\mdfdefinestyle{enviStyle}{
	innertopmargin = 10pt,
	linewidth      = 1pt,
	frametitlerule = true,
	roundcorner    = 2pt%
}


\newenvironment{CountingDefinition}[2][]{%
	\ifstrempty{#1}%
	{\mdfsetup{%
			frametitle={{\strut ~}}}
	}%
	{\mdfsetup{%
			frametitle={{\strut ~#1}}}%
	}%
	\mdfsetup{
		nobreak                   = true,
		linecolor                 = OliveGreen,
		frametitlebackgroundcolor = OliveGreen!50,
		style                     = enviStyle
	}
	\begin{mdframed}[]\relax%
		\label{#2}}{\end{mdframed}}



%\renewcommand{\labelitemi}{$\textcolor{OliveGreen}{\bullet}$}
\renewcommand{\labelitemii}{$\textcolor{Black}{\circ}$}
\renewcommand{\labelitemiii}{$\textcolor{Black}{\cdot}$}
%\renewcommand{\labelitemiv}{$\textcolor{OliveGreen}{\ast}$}
\usepackage[htt]{hyphenat}

\newcommand{\heart}{\ensuremath\heartsuit}

%----------------------------------------
% Rounded corners in graphics with \cutpic
\newsavebox{\picbox}	
\newcommand{\cutpic}[3]{
	\savebox{\picbox}{\includegraphics[width=#2]{#3}}
	\tikz\node [draw, rounded corners=#1, line width=4pt,
	color=white, minimum width=\wd\picbox,
	minimum height=\ht\picbox, path picture={
		\node at (path picture bounding box.center) {
			\usebox{\picbox}};
	}] {};}

% Checkbox
\usepackage{pifont}
\newcommand{\cmark}{\ding{51}}%
\newcommand{\done}{\rlap{$\square$}{\raisebox{2pt}{\large\hspace{1pt}\cmark}}%
	\hspace{-2.5pt}}


%----------------------------------------
% References
%\usepackage[backend=biber, maxbibnames=99]{biblatex}
%\addbibresource{references.bib}
%----------------------------------------
%\ohead{\\
%		Pina Kolling\\
%		piko0011}
	
%\usepackage[vmargin=1in,hmargin=1in]{geometry}
%\usepackage{enumitem}
%\setlist[itemize]{topsep=0pt,before=\leavevmode\vspace{-1.5em}}
%\setlist[description]{style=nextline}

%----------------------------------------------------------------------------
%----------------------------------------------------------------------------

\begin{document}
	
	\title{\color{OliveGreen} \vspace{-5ex} Weekly Diary}
	\author{
		Master thesis course in Computing Science \\
		\textbf{Pina Kolling}
	}
	\date{\vspace{-5ex}}
	
	\maketitle
	 
	\begin{table}[H]
		\renewcommand\arraystretch{1.4}\arrayrulecolor{OliveGreen}
		\begin{longtable}{@{\,}r <{\hskip 2pt} !{\foo} >{\raggedright\arraybackslash}p{12cm}}
			% \toprule
			% \addlinespace[1.5ex]
			Week \ 3 & Introduction and first work on project plan \\
			Week \ 4 & Finish project plan, start setting up code on my computer  \\
			Week \ 5 & First research on the topic, including finding literature, set up Git and \LaTeX \ for master thesis (on work laptop, my laptop and stationary pc), document execution of code \\
			Week \ 6 & Set up code on my computer and first familiarizing with codebase, finding literature, document execution of code \\
			Week \ 7 & Researching options of melt framework (implementing, documenting the process and literature research)  \\
			Week \ 8 & Implementing, documenting the process and literature research and vacation with my grandmother (she turns 90 \heart) so probably reduced work capacity \\
			Week \ 9 & Implementing, documenting the process and literature research \\
			Week 10 & Implementing, documenting the process and literature research and creating slides for the midterm seminar, evaluating if it is possible to obtain colour-corrected video results using JIT and then specify or readjust the focus \\
			Week 11 & Implementing, documenting the process and literature research, midterm seminar \\
			Week 12 & Implementing, documenting the process and literature research, search or implement offline colour correction software and other suitable solutions for comparison (if needed) \\
			Week 13 &  Implementing, documenting the process and literature research \\
			Week 14 &  Implementing, documenting the process and literature research \\
			Week 15 &  Writing \\
			Week 16 &  Writing \\
			Week 17 &  Writing \\
			Week 18 &  Writing \\
			Week 19 &  Finalizing, reworking and applying feedback \\
			Week 20 &  Hand in final version of the thesis \\
			Week 21 &  Create Slides for the thesis seminar \\
			Week 22 &  Thesis seminar (defence and opposition) \\
			Week 23 &  Opponent thesis report \\
		\end{longtable}
	\end{table}


\newpage



	\section*{Week 3}
	
	\begin{description}
		








%-----------------------------------------------------------------------------		
		\item[16.01.24, Tuesday] 
		
		\begin{itemize}
			\item[]
			\item First meeting at university
		\end{itemize}








%-----------------------------------------------------------------------------		
		\item[17.01.24, Wednesday]
		
		\begin{itemize}
			\item[]
			\item Setting up file and git for weekly diary
			\item Writing first mail with topic specification to Vicen\c{c} Torra
			\item Keeping my supervisor at Codemill (Urban Söderberg) in the loop
			\item Begin with project plan (setting up the file, etc.)
		\end{itemize}








%-----------------------------------------------------------------------------		
		\item[18.01.24, Thursday]
		
		\begin{itemize}
			\item[]
			\item Getting a supervisor from university assigned (Cem Okulmus) % and first communication with him
			\item Continue work on project plan:
			\begin{itemize}
				\item Introduction
			\end{itemize}
			
			\item First research on:
			\begin{itemize}
				\item Just-In-Time (JIT), WebRTC, h.264, Melt framework
				\item Infrastructure model of the system
			\end{itemize}
		\end{itemize}







%-----------------------------------------------------------------------------		
\item[19.01.24, Friday]

\begin{itemize}
	\item[]
				\item Continue work on project plan:
	\begin{itemize}
		\item Problem formulation
		\item Method
		\item Infrastructure model
	\end{itemize}
\end{itemize}







%-----------------------------------------------------------------------------		
\item[20.01.24, Saturday]

\begin{itemize}
	\item[]
	\item Continue work on project plan:
	\begin{itemize}
		\item Evaluation methods
		\item Self assessment
	\end{itemize}
	\item Looking into previous master thesis that was written at Codemill
\end{itemize}

	\end{description}








%-----------------------------------------------------------------------------	
%-----------------------------------------------------------------------------	
\newpage
\section*{Week 4}

Info: The Codemill logo marks the days at which I have been at the company's office.

\begin{description}


%-----------------------------------------------------------------------------		
\item[22.01.24, Monday]
\begin{itemize}
	\item[]
	\item Set up git on other computer
		\item Continue work on project plan:
	\begin{itemize}
		\item Resources
		\item Read again and correct
		\item Deciding on a title
	\end{itemize}
	\item Send projectplan to supervisor at Codemill (Urban Söderberg)
	\item Send projectplan to supervisor at university (Cem Okulmus)
\end{itemize}







%-----------------------------------------------------------------------------		
\item[23.01.24, Tuesday]
\begin{itemize}
	\item[]
	\item First meeting with supervisor at university (Cem Okulmus)
	\item Rework and additional info on project plan:
	\begin{itemize}
		\item Change JIT definition
		\item Add timeline
		\item Add challenges
	\end{itemize}
	\item Add timeline weekly diary and adapt setup of weekly diary (counting in calendar weeks)
\end{itemize}








%-----------------------------------------------------------------------------		
\item[24.01.24, Wednesday]
\begin{itemize}
	\item[]
	\item Prepare laptop to set up code on it
\end{itemize}















%-----------------------------------------------------------------------------		
\item[25.01.24, Thursday]
\includegraphics[width=1.4cm]{codemill.png}
\begin{itemize}
	\item Setting up the code on my laptop at Codemill 
	
	(generating ssh key, cloning git repositories, installing node.js and docker, etc)
	\begin{itemize}
		\item Problem: My RAM was not sufficient and the code could not be executed
		\item Solution: Looking for a company laptop to execute the code
	\end{itemize}
\end{itemize}


















%-----------------------------------------------------------------------------		
\item[26.01.24, Friday]
\includegraphics[width=1.4cm]{codemill.png}
\begin{itemize}
	\item Setting up the code on the new laptop at Codemill
	\begin{itemize}
		\item Problem: Space in user name on the device which makes some paths not working
		\item Solution: Setting up windows with a new user (to do)
		\item Info: The code has not been run on a windows system before
	\end{itemize}
\end{itemize}

	\end{description}









%-----------------------------------------------------------------------------	
%-----------------------------------------------------------------------------	
\newpage
\section*{Week 5}

\begin{description}
	
	
%-----------------------------------------------------------------------------		
\item[29.01.24, Monday]
	\begin{itemize}
		\item[]
		\item Being sick \frownie{}
	\end{itemize}
	
	
	
	
	
	
	
%-----------------------------------------------------------------------------		
\item[30.01.24, Tuesday]
	\begin{itemize}
		\item[]
		\item Being sick \frownie{}
	\end{itemize}





%-----------------------------------------------------------------------------		
\item[31.01.24, Wednesday]
	\begin{itemize}
		\item[]
		\item Being sick \frownie{}
		\item Setting up new windows user
		\item Setting up code on new laptop (frontend running but problems with backend/docker container)
		\item Document execution of code:
	\end{itemize}




\end{description}
\vspace*{-0.8em}
\begin{CountingDefinition}[Setting up the code]{def:codesetup}
	\vspace*{-0.2em}
	\begin{itemize}
		\item Generate ssh key (\texttt{ssh-keygen}) and add to GitLab
		\item Clone git repositories (jit-webrtc and accurate-player-3-core)
		\item Install node.js and set path variables for npm (and yarn)
		\item Install and run docker
		\item Execute jit-webrtc code with command from README with \texttt{docker/main/main.sh --threads 16 --port 8080 \$VIDEOFILE} (not working!)
		\item Execute accurate player code (run \texttt{npm install --force}, \texttt{npm install yarn} and then npm start, resolve errors, fix dependenciy problems with \texttt{npm audit fix --force} (potentially twice))
	\end{itemize}
	
\end{CountingDefinition}

\begin{description}





%-----------------------------------------------------------------------------		
\item[01.02.24, Thursday]
\begin{itemize}
	\item[]
	\item Being sick \frownie{}
	\item Installing slack
	\item Looking into the backend/docker problem %(\texttt{docker: Error response from daemon: error gathering device information while adding custom device "C": not a device node.})
	\item Setting up WeeklyDiary git and tex file on Codemill-laptop
\end{itemize}






%-----------------------------------------------------------------------------		
\item[02.02.24, Friday]
\begin{itemize}
	\item[]
	\item Being sick \frownie{}
	\item Trying to solve the docker/backend problem (still unsolved)
	\item Setting up git and tex file for master thesis on stationary PC
	\item Creating title page
	\item Structure for thesis
	\item First research and adding of references
	\item First writing in introduction
\end{itemize}




%-----------------------------------------------------------------------------		
\item[03.02.24, Saturday]
\begin{itemize}
	\item[]
	\item Being sick \frownie{}
	\item Trying to solve the docker/backend problem (still unsolved):
	\begin{itemize}
		\item Inspecting \texttt{main.sh} script file
		\item Inspecting docker problems regarding windows
		\item \texttt{docker-run.sh} not found or opened... Changing the path does not seem to help and the file does exist (feedback: no such file or directory)
		\item Setting up python
	\end{itemize}
\end{itemize}


\end{description}








%-----------------------------------------------------------------------------	
%-----------------------------------------------------------------------------	
\newpage
\section*{Week 6}

\begin{description}

%-----------------------------------------------------------------------------		
\item[05.02.24, Monday]
\includegraphics[width=1.4cm]{codemill.png}
\begin{itemize}
	\item Run backend/docker (finally!): 
	\begin{itemize}
		\item Make changes in \texttt{main.sh} (last line): remove \texttt{--device /dev/fuse} and change path to  \texttt{//opt/jit-webrtc/jit/docker-run.sh}
	\end{itemize}
	\item Problem: Connectivity issues between browser and docker
	\item Solution: Installing Linux and not running it under Windows
\end{itemize}









%-----------------------------------------------------------------------------		
\item[06.02.24, Tuesday]
\begin{itemize}
	\item[]
	\item Installing Linux Ubuntu 22.04 (not booting after updates)
	\item Installing Linux Ubuntu 23.10 (does not work at all)
	\item Researching and writing an introduction about Codemill
	\item Installing Linux Ubuntu 22.04
	\begin{itemize}
		\item The problem originated from the NVIDIA graphics card. Before updating, the drivers had to be installed with \texttt{sudo ubuntu-drivers autoinstall}.
	\end{itemize}
	\item Installing docker, node.js, git, miktex, texstudio and cloning repositories
	\item Adding to weekly diary: Codemill logo for each day I was at the company's location
	\item Executing frontend
	\item Executing backend in docker container
\end{itemize}









%-----------------------------------------------------------------------------		
\item[07.02.24, Wednesday]
\includegraphics[width=1.4cm]{codemill.png}
\begin{itemize}
	\item Connecting backend and frontend
	\item Running the code
	\item Setting docker timeout from 15s to 150s in \texttt{main.py}
	\item Create private git repositories to store work progress 
	\item Research on WebRTC and transcoding and looking into code of JIT-WebRTC
	\item Adding labels and references to structure of master thesis tex file
	\item Adding README files of code base to master thesis tex file
\end{itemize}
\end{description}



\vspace*{-0.8em}
\begin{CountingDefinition}[Running the code] {def:coderun}
\vspace*{-0.2em}
\begin{itemize}
	\item Frontend: 
	\begin{itemize}
		\item Open folder \texttt{accurate-player-3-core/packages/demo} in terminal
		\item Execute \texttt{JIT\_BACKEND=http://localhost:8080 yarn start} or \texttt{./start.sh}
	\end{itemize}
	\item Backend:
	\begin{itemize}
		\item Open folder \texttt{jit-webrtc} in terminal
		\item Execute \texttt{docker/main/main.sh --threads 16 --port 8080 https://s3.eu-central-1.amazonaws.com/accurate-player-demo-assets/timecode/sintel-2048-timecode-stereo.mp4
		}
	\end{itemize}
	\item Open \url{http://localhost:5000/controls/jit/index.html} in browser
\end{itemize}

\end{CountingDefinition}

\begin{description}










%-----------------------------------------------------------------------------		
\item[08.02.24, Thursday]
% \includegraphics[width=1.4cm]{codemill.png}
\begin{itemize}
	\item[]
	\item Looking into the backend code, README and the system's components, summarizing and taking notes in the thesis file:
	\begin{itemize}
		\item Audio Video Interleave (AVI)
		\item Named pipe
		\item Create diagram of system
		\item Python documentation
		\item Web services and REST API
	\end{itemize}
	\item Structure of the thesis
\end{itemize}














%-----------------------------------------------------------------------------		
\item[09.02.24, Friday]
% \includegraphics[width=1.4cm]{codemill.png}
\begin{itemize}
	\item[]
	\item Looking into the backend code and the system's components, summarizing and taking notes in the thesis file:
	\begin{itemize}
		\item Docker container
	\end{itemize}
	\item Looking into the frontend code and README, summarizing and taking notes in the thesis file:
	\begin{itemize}
		\item Node.js, yarn and npm
	\end{itemize}
\end{itemize}



%-----------------------------------------------------------------------------			

		
		
\end{description}
	
	
	
	







%-----------------------------------------------------------------------------	
%-----------------------------------------------------------------------------	
\newpage
\section*{Week 7}	





\begin{description}
	
	
	
	
	
	%-----------------------------------------------------------------------------		
	\item[12.02.24, Monday]
	\includegraphics[width=1.4cm]{codemill.png}
	\begin{itemize}
		\item Write to do list for the next steps and update time schedule
		\item Look into MLT FX and integration of OpenGL and GLSL?
		\item Looking into suitable filters (aka plugins) in melt
		\begin{itemize}
			
			\item Maybe suitable: \texttt{avfilter.colorbalance}, \texttt{avfilter.colorchannelmixer}, \newline \texttt{avfilter.colorcontrast}, \texttt{avfilter.colorlevels}, \newline \texttt{avfilter.colortemperature}, \texttt{frei0r.coloradj\_RGB}, \texttt{frei0r.colorize}
			
			
			\item Probably not suitable: \texttt{avfilter.colorcorrect}, \texttt{avfilter.colorhold}, \newline \texttt{avfilter.colorize}, \texttt{avfilter.colorkey}, \texttt{avfilter.colormatrix}, \newline \texttt{avfilter.colorspace}, \texttt{frei0r.colordistance}, \texttt{frei0r.colorhalftone}, \newline  \texttt{frei0r.colortap}, \texttt{frei0r.three\_point\_balance}, \texttt{frei0r.contrast0r}, \texttt{tcolor}
			
		\end{itemize}
	\end{itemize}
	

\end{description}	


% TODO
% ⚪️ literature research
% ⚪️ einleitung umstrukturieren und mehr auf farben beziehen
% ⚪️ look into code base
% ⚪️ mail an uni supervisor
% ⚪️ Abbildungs- und Tabellenverzeichnis
% Does melt already have an option for this?
	
	




\vspace*{-0.8em}
\begin{CountingDefinition}[To Do]{def:todo}
	\vspace*{-0.2em}
	
	\begin{itemize}
		\item Figure out, where the colour grading should be implemented
		\begin{enumerate}
			\item Does melt already have an option for this?
			\item Can it maybe only be done when the video is paused?
			\item Is there a different place in the system, where the colour grading can be done?
		\end{enumerate}
	\end{itemize}
	
\end{CountingDefinition}







\begin{description}
	
	
	\item[13.02.24, Tuesday]
	\includegraphics[width=1.4cm]{codemill.png}
	\begin{itemize}
		\item Looking briefly into \texttt{melt.c}, \texttt{JitControl.proto}, \texttt{JitStatus.proto} and other melt files to find out, where to attach a filter/plugin to a video and where the quality setting is changed
		% \item Looking briefly into proto 3 syntax
		\item Getting first overview over structure of melt
		\item Execute melt with filter without JIT to test the filters: \texttt{melt https://s3.eu-central-1.amazonaws.com/accurate-player-demo-assets/timecode/sintel-2048-timecode-stereo.mp4 -filter avfilter.colorbalance av.rs=1 av.gm=1 av.bh=1} for intense colours
	\end{itemize}


\begin{minipage}{0.5\textwidth}
	\cutpic{0.3cm}{0.9\textwidth}{colourdefault.png}
	Original colours
\end{minipage}\begin{minipage}{0.5\textwidth}
	\cutpic{0.3cm}{0.9\textwidth}{colourhigh.png}
	Colours with \texttt{av.rs=1} \texttt{av.gm=1} \texttt{av.bh=1}
\end{minipage}

\begin{itemize}
	\item[$\rightarrow$] This can be used for the offline comparison
	\item Looking into \texttt{local\_melt.py} and \texttt{main.py}
\end{itemize}

	




\item[14.02.24, Wednesday]
\includegraphics[width=1.4cm]{codemill.png}
\begin{itemize}
	\item Looking into \texttt{local\_melt.py} and \texttt{main.py} and trying to add command for melt there
	\item On the trail of print statements - disappeared or user error? (Aka figuring out where the logging info and print statements are printed to get more insight of the code)
	\item[$\rightarrow$] Add \texttt{-v} to command to see logs in the terminal:
	\texttt{docker/main/main.sh -v --threads 16 --port 8080 https://s3.eu-central-1.amazonaws.com/ accurate-player-demo-assets/timecode/sintel-2048-timecode-stereo.mp4
	}
	\item Fun fact (might be useful later): Find the name of your docker container with \texttt{docker ps} and get info with \texttt{docker logs --follow <container-name>} (but the missing info cannot be found there either so far)
	\item Changing overall input command for melt and adding a video filter to the video, that can be viewed in the accurate player using JIT \includegraphics[height=0.32cm]{success.png}\\
	\cutpic{0.2cm}{0.9\textwidth}{ap_red.png}
	% \includegraphics[width=0.9\textwidth]{ap_red.png}
	\item Add filter to \texttt{melt\_args} in different place in code
	% \item To Do: Can it be changed? Add slider in frontend with JSON, etc
	\item Start looking into main.py and frontend code base to implement slider to change colour intensity and add different options from the frontend
\end{itemize}





\item[16.02.24, Friday]	
\begin{itemize}
	\item[]
	\item Look into frontend code
	\item Add slider for colour red in \texttt{packages/demo/src/controls/jit/index.html} (without any backend functionality so far): \\
	\includegraphics[width=5.9cm]{slider_red2.png}
	\item Small \LaTeX \ \texttt{tikz} side project for rounded corners of the graphics in my documents: \\
	\begin{tikzpicture} 
		\begin{scope}
			\clip [rounded corners=.3cm] (0,0) rectangle coordinate (centerpoint) (5.9,1.6cm); 
			\node [inner sep=0pt] at (centerpoint) {\includegraphics[width=6.0cm]{slider_red2.png}}; 
		\end{scope}
	\end{tikzpicture}
	
\end{itemize}
	
	
	
	
\item[18.02.24, Sunday]	
\begin{itemize}
	\item[]
	\item Downloading ressources to prepare offline work during travel
	\item Trying to get docker running with local video files (problem with \texttt{google-crc32c})
	
\end{itemize}
	
	
	
	
\end{description}	



	









%-----------------------------------------------------------------------------	
%-----------------------------------------------------------------------------	
\newpage
\section*{Week 8}	





\begin{description}
	
	
	
%----------------------------------------------------------------------------
\item[19.02.24, Monday]
\begin{itemize}
	\item[]
	\item Reading the accurate player code and following the data flow of the input of the quality slider, especially in \\ \texttt{packages/demo/src/controls/jit/index.html}, \\ \texttt{packages/demo/src/controls/jit/JITDemo.ts}, \\ \texttt{packages/jit/dist/index.d.ts}, \\ \texttt{packages/jit/src/JITService.ts}, \\ \texttt{packages/demo/src/controls/jit-session/JITSessionDemo.ts} and \\ \texttt{packages/core/dist/index.d.ts}
	
\end{itemize}
	
	
	
	
	
%----------------------------------------------------------------------------
	\item[23.02.24, Friday]
	\begin{itemize}
		\item[]
		\item Implement feedback in the  thesis file:
		\begin{itemize}
			\item \textit{Related work} before \textit{Structure}
			\item Merging Chapter 2 and 3 into a \textit{Preliminaries} Chapter
			\item Merge the Subsections in Chapter 2 and 3
			\item Remove Subsection \textit{Implementation Details} (Chapter 5)
			\item Rename Chapter 5 to \textit{Experimental Evaluation and Discussion}
			\item List of Tables and List of Figures
			\item References: Use footnotes for the URLs (f.ex. Codemill Website) and remove them from reference list:
			\begin{itemize}
				\item Define command \texttt{$\backslash$myfootcite}
			\end{itemize}
		\end{itemize}
		\item Add caption and short caption for list of figures to each graphic
		\item Define command \texttt{$\backslash$cutpic} for rounded corners in graphics and apply this to the thesis file and the weekly diary
		\begin{itemize}
			\item[$\rightarrow$] Because of very restricted internet, the work on the implementation of the colour correction cannot really be continued (code cannot be executed here) and this is why less important tasks like design of the thesis are done now
		\end{itemize}
		\item Seperate multiple footnote citations with a comma (using \texttt{$\backslash$textsuperscript$\{$,$\}$} to maintain the correct font of the comma)
		\item Write on thesis: Motivation and Research Questions
	\end{itemize}

% TODO:
% Code muss funktionieren lol
% README References?
% motivation colour sources
% motivation: add location or source of photo?
% add how many percent of people have rot grün schwäche or other colour issues
\end{description}	













%-----------------------------------------------------------------------------	
%-----------------------------------------------------------------------------	
\newpage
\section*{Week 9}	





\begin{description}



%----------------------------------------------------------------------------
\item[28.02.24, Wednesday]
\begin{itemize}
	\item[]
	\item Working on implementation -- problems to run the code 
	\item[$\rightarrow$] Suspecting my mother's weird and bad internet setup to be the cause of the problem and hoping, that it will run when I return to Sweden tomorrow. \\
	(Fun fact: Germany has very bad internet.)
\end{itemize}



%----------------------------------------------------------------------------
\item[29.02.24, Thursday]
\begin{itemize}
	\item[]
	\item Looking into the data path of the quality parameter in the accurate player code for better understanding:
	\begin{itemize}
		\item[$\rightarrow$] \texttt{quality-slider} in \texttt{packages/demo/src/controls/jit/index.html} gets input
		
		\item[$\rightarrow$] input value is read out in \texttt{packages/demo/src/controls/jit/JITDemo.ts} and then \texttt{player?.api.setQuality(value)} is called
		
		\item[$\rightarrow$] \texttt{player} has type \texttt{JITPlayer} and \texttt{api} has type \texttt{JitService}
		
		\item[$\rightarrow$] class \texttt{JitService} extends class \texttt{Service} 
	\end{itemize}
\end{itemize}











%----------------------------------------------------------------------------
\item[01.03.24, Friday]
\includegraphics[width=1.4cm]{codemill.png}
\begin{itemize}
	\item Making some changes in the schedule and informing supervisors via mail
	\item Adding red value in different places in the code and test the results for better understanding:
	\begin{itemize}
		\item Backend: Adding red value with \texttt{"av.rs=$\%$.1f" $\%$ kwargs['av.rs']} to have a variable that can (hopefully) be changed from the frontend and then the red value will maybe update when it is changes (in \texttt{jit-webrtc/jit/local\_melt.py})
		\item Adding \texttt{RED\_VALUE = 1} and \texttt{$^{**}$({"av.rs": RED\_VALUE})} in \texttt{jit-webrtc/jit/main.py} as a parameter to \texttt{execute\_local\_melt} so it is contained in \texttt{kwargs[$'$av.rs$'$]} in \texttt{jit-webrtc/jit/local\_melt.py}
		\item Add value for \texttt{rs} in \texttt{jit-webrtc/jit/ffprobe.py}
		\item[$\rightarrow$] This seems to be the red value with which the video gets send initially from melt 
	\end{itemize}
	\item Analysing line $\sim$180-200 in \texttt{jit-webrtc/jit/main.py} to add the red value there for it to be able to be updated:
	\begin{itemize}
		\item How do I get \texttt{encoder = videosender.\_RTCRtpSender\_\_encoder} to return the red value? 
		\item Adding \textit{red} in line $\sim$180-200
		\item Adding \textit{red} in line $\sim$530
    \end{itemize}
\end{itemize}

\end{description}	













%-----------------------------------------------------------------------------	
%-----------------------------------------------------------------------------	
\newpage
\section*{Week 10}	





\begin{description}



%----------------------------------------------------------------------------
\item[04.03.24, Monday]
\includegraphics[width=1.4cm]{codemill.png}
\begin{itemize}
	\item Red value can be changes when starting docker with \texttt{--rs <value>}
	\item \texttt{json.dumps()} function will convert a subset of Python objects into a json string
	\begin{itemize}
		\item[$\rightarrow$] add red value to be send to melt in line $\sim$180-200 in \texttt{jit-webrtc/jit/main.py}
		\item[$\rightarrow$] retrieve red value from frontend in line $\sim$180-200 in \texttt{jit-webrtc/jit/main.py}
 	\end{itemize}
	\item Where does the data get send to melt? Where does melt receive it? Where is it processed?
	\begin{itemize}
		\item \texttt{local\_melt.py}? $\rightarrow$ Seems to be for initialization
		\item Reading through \texttt{README}s in subfolders:
		
		\vspace*{-0.5em}
		\begin{CountingDefinition}[Data Flow]{1}
			
			Commands are sent from the browser to \texttt{main.py} via a WebRTC data channel,
			and then from \texttt{main.py} to \texttt{melt} via a \texttt{stdout/stdin}.
			
			\texttt{melt} in turn sends an AVI stream with rendered video and audio to \texttt{main.py} via a named pipe,
			and status messages and metadata is sent to main.py via another named pipe.
		\end{CountingDefinition}
		\item Looking into \texttt{class StdinVideoTrack(VideoStreamTrack):} in line 393 in \texttt{main.py}
		\item Trying to send input of quality slider to main.py:
		\begin{itemize}
			\item Adding \texttt{getRedSliderValue()} and more in \texttt{packages/demo/src/controls/jit-session/JITSessionDemo.ts}
			\item Adding a bunch of code in the frontend
		\end{itemize}
		% \item Session? 
		% accurate-player-3-core/packages/controls/src/components/quality-select/index.js ???
	\end{itemize}
\end{itemize}













%----------------------------------------------------------------------------
\item[05.03.24, Tuesday]
\includegraphics[width=1.4cm]{codemill.png}
\begin{itemize}
	\item Sending input value from slider to backend (to \texttt{main.py})
	\item Solving compile issues (using \texttt{yarn start} in \texttt{jit}-folder and \texttt{./start.sh} in \texttt{demo}-folder)
	\item Value for red included in JSON, that the backend receives but the value is always 1?
	%\begin{itemize}
		% \item \texttt{JitService.ts} $\rightarrow$ \texttt{setRedValue()}
		% \item Error in \texttt{JitService.ts} in line 696 (from 605 on)
		% \item JSON object posted in \texttt{JitService.ts} from 605 on (?)
	%\end{itemize}
\end{itemize}










%----------------------------------------------------------------------------
\item[07.03.24, Thursday]
% \includegraphics[width=1.4cm]{codemill.png}
\begin{itemize}
	\item[]
	\item Why does the code not run anymore?
	\begin{itemize}
		\item[$\rightarrow$] Fixed
	\end{itemize}
	\item Why is the value of \texttt{red} when being sent to the backend always 1?
		\begin{itemize}
		\item[$\rightarrow$] Overwritten with -1 now in \texttt{main.py} but where was the 1 coming from and how to retrieve the value from the slider from the encoder?
	\end{itemize}

\end{itemize}











%----------------------------------------------------------------------------
\item[08.03.24, Friday]
\includegraphics[width=1.4cm]{codemill.png}
\begin{itemize}
	\item Value from frontend slider retrieved in \texttt{main.py}
\end{itemize}



	
	
\end{description}	















%-----------------------------------------------------------------------------	
%-----------------------------------------------------------------------------	
\newpage
\section*{Week 11}	





\begin{description}
	
	
	
	%----------------------------------------------------------------------------
	\item[11.03.24, Monday]
	\begin{itemize}
		\item[]
		\item Organisational tasks: Plan Midterm Seminar, prepare DataTjej event, plan next steps for implementation
		\item Solving Git issues, which resulted into VSC issues that needed to be solved too
		\item Making slides for Midterm Seminar
		\item Trying to add red value from slider in \texttt{main.py} in line 813: Is this for the initialization only?
	\end{itemize}
	
	
	
	
	
	
	
	
	
	
	%----------------------------------------------------------------------------
	\item[12.03.24, Tuesday]
	\includegraphics[width=1.4cm]{codemill.png}
	\begin{itemize}
		\item Red value in \texttt{main.py} in line 813 for the initialization with \texttt{--rs} for red value in terminal
		\item \texttt{flush\_control} line 90-95 in \texttt{main.py} seems to send the command message to the melt process (play, pause, etc.)
		\item \texttt{flush\_command\_fifo}: Parse command json from browser, send command messages to melt process
		\item Line 177-210 in \texttt{main.py}: WebRTC data channel (dc) is being utilized to send data to the Melt backend.
		\begin{itemize}
			\item[$\rightarrow$] Race condition or synchronization issue, because the value of red from the frontend is never assigned there
			\item[$\rightarrow$] Sending data on a different place to melt? 
		\end{itemize}
	\end{itemize}
	
	
	
	
		
	
	
	
	
	
	
	
	%----------------------------------------------------------------------------
	\item[13.03.24, Wednesday]
	\includegraphics[width=1.4cm]{codemill.png}
	\begin{itemize}
		\item \texttt{json.dumps} in line 189 in \texttt{main.py}
		\item Trying to get red value from \texttt{main.py} to \texttt{ffprobe.py}
		\begin{itemize}
			\item[$\rightarrow$] Is this only relevant for initialization?
			\item[$\rightarrow$] Add parameter for red value in \texttt{def probe(url, rs = None)}
		\end{itemize}
		\item \textit{Practising} presentation for mid term seminar and adapting the slides
	\end{itemize}



	
%----------------------------------------------------------------------------
\item[14.03.24, Thursday]
\includegraphics[width=1.4cm]{codemill.png}
\begin{itemize}
	\item Mid term seminar
	\item DataTjej event at Codemill
\end{itemize}





	
%----------------------------------------------------------------------------
\item[16.03.24, Saturday]
\begin{itemize}
	\item[]
	\item \texttt{ffprobe.py} seems to only be called in initialization and the value for red is not assigned correctly in \texttt{main.py} $\rightarrow$ asynch/race condition
\end{itemize}
	
	





%----------------------------------------------------------------------------
\item[17.03.24, Sunday]
\begin{itemize}
	\item[]
	\item Researching runtime manipulation of melt command more in depth
\end{itemize}




	
	

	
	
\end{description}	











%-----------------------------------------------------------------------------	
%-----------------------------------------------------------------------------	
\newpage
\section*{Week 12}	





\begin{description}
	
	
	
	%----------------------------------------------------------------------------
	\item[19.03.24, Tuesday]
	\includegraphics[width=1.4cm]{codemill.png}
	\begin{itemize}
		\item Asking for help at Codemill and making new to do list
		\begin{itemize}
			
			\item[\done] Add red value in \texttt{Jit.Control.proto} and adding message to retrieve specific value there
			
			\item[\done] \texttt{main.py}: Add command for red value in \texttt{flush\_command\_fifo} ($\sim$ line 140)
			
			\item[\done] Frontend: Send red value in different spot to receive in \texttt{flush\_command\_fifo}
			
			\begin{itemize}
				\item[\done] Add red value to JitAction in \texttt{JITService.ts} ($\sim$ line 30)
				
				\item[\done] Solve issue: \texttt{AttributeError:} \texttt{module} \texttt{'JitControl\_pb2'} \texttt{has} \texttt{no} \texttt{attribute} \texttt{'RED\_VALUE'}
				
				\begin{itemize}
					\item Install protoc
					\item Navigate to \texttt{/jit-webrtc/melt}
					\item Execute \texttt{protoc --c\_out=. JitStatus.proto} and \\ \texttt{protoc --c\_out=. JitControl.proto} \\ to compile the proto files
					\item Execute \texttt{protoc --python\_out=. JitStatus.proto} and \\ \texttt{protoc --python\_out=. JitControl.proto} \\ to generate the python files
					\item Wrong melt folder? Wrong proto version in code? Where does this come from? Get different version...
					\item Changing from proto3 to proto2 and downloading correct files
					\item Pulling \texttt{https://github.com/sirf/mlt} and then executing \texttt{jit-webrtc/jit\$ protoc  --python\_out=.  -I../../mlt/src/melt/  JitControl.proto} and \texttt{jit-webrtc/jit\$ protoc  --python\_out=.  -I../../mlt/src/melt/  JitStatus.proto}
					\item[$\rightarrow$] Found bug in code base: \texttt{KEEP\_ALIVE} missing in the current version of the \texttt{.proto} file -- bug was fixed now
				\end{itemize}
				
			\end{itemize}
				
			\item[\done] \texttt{main.py}: Retrieve red value from JSON in \texttt{flush\_command\_fifo} ($\sim$ line 140)
			
			\item[$\square$] Apply filter in melt (how exactly?)
			
		\end{itemize}
		% Fragen strukturieren:
		% - Where should I send the data?
		% - Value never assigned correctly there
		% - Where does it get received? Use mlt functions like      	\texttt{mlt_service_attach} or where should the filter be applied?
	\end{itemize}
	
	
	
	
	
	
		
	
%----------------------------------------------------------------------------
	\item[20.03.24, Wednesday]
	\includegraphics[width=1.4cm]{codemill.png}
	\begin{itemize}
		
			\item Cleaning up the code a little
			
			
			\item[\done] Figuring out where/how to apply filter in melt:
			
			\begin{itemize}
				
					
				\item How/Where are the other commands applied? (\texttt{PLAY}, \texttt{KEEP\_ALIVE}, etc )
				
				
				\item Where does \texttt{flush\_control} send the commands to?
				\begin{itemize}
					\item \texttt{melt\_server\_socket\_address = LOCAL\_MELT["melt-ssock-addr"]}
					\item Server socket: \texttt{/tmp/melt-sock-7}
				\end{itemize}
			
				\item Red value needs to used in a function that attaches a filter $\rightarrow$ Which function? Possibly \texttt{mlt\_filter\_connect} or \texttt{mlt\_service\_attach}
				
			\end{itemize}	
		
			\item[$\square$] Adding code in  \texttt{jit-webrtc/mlt/src/melt/jit.c} in the \texttt{jit\_action} function (wip)
		
		

			
	\end{itemize}

	
	
		

%----------------------------------------------------------------------------
\item[21.03.24, Thursday]
\begin{itemize}
	
	\item[]
	
	\item[\done] Adding code in  \texttt{jit-webrtc/mlt/src/melt/jit.c} in the \texttt{jit\_action} function for \texttt{case CONTROL\_TYPE\_\_RED\_VALUE}
	
	\item[$\rightarrow$] Able to apply grayscale filter correctly but not \texttt{avfilter.colorbalance}
	
	\begin{itemize}
		\item Looking into implementation of filter
	\end{itemize}


	\item \fbox{%
		\parbox{\linewidth}{%
			Changing the red value with the slider works while the video is playing! YAY!
		}%
	}
	
	\item TO DO LIST:
	
	\begin{itemize}
		\item[$\square$] Adding all of this for green and blue too
		\item[$\square$] Seems to work immediately - comparing with offline version still suitable?
		\item[$\square$] Describing functionality of MLT framework in master thesis
	\end{itemize}
	
	
\end{itemize}

	
	
	
	
	
	
	
	
	
\end{description}	


	
	
	
	
\end{document}